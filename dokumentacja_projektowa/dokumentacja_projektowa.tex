\documentclass[12pt,a4paper,oneside]{article}

\usepackage[QX]{polski}

\usepackage[utf8]{inputenc}
\usepackage{latexsym}
\usepackage{tgpagella}
\usepackage{lmodern}
\usepackage{amsmath,amsthm,amsfonts,amssymb,alltt}
\usepackage{epsfig}
\usepackage{pdflscape}
\usepackage{caption}
\usepackage{indentfirst}
\usepackage{float}
%\usepackage{showkeys}
\bibliographystyle{plabbrv}


\usepackage{color}
\usepackage[polish]{babel}
\usepackage{datetime2}
\usepackage[x11names,dvipsnames,table]{xcolor}
\usepackage{hyperref}
\hypersetup{
pdfauthor={Roman Czapla, Olaf Bar},
colorlinks=True,
linkcolor=darkgray,  % color of internal links (change box color with linkbordercolor)
citecolor=BrickRed,  % color of links to bibliography
filecolor=Magenta,   % color of file links
urlcolor=BlueViolet}	%%pdfpagemode=FullScreen}

% diagramy, grafy itp.
\usepackage{tikz}
\usetikzlibrary{positioning}
\usetikzlibrary{arrows}
\usetikzlibrary{arrows.meta}
\usetikzlibrary{chains,fit,shapes,calc}
\tikzset{main node/.style={circle,fill=blue!20,draw,minimum size=1cm,inner sep=0pt}}

% algorytmy
\usepackage[linesnumbered,lined,commentsnumbered]{algorithm2e}
\SetKwFor{ForEach}{for each}{do}{end for}%
\SetKwFor{ForAll}{for all}{do}{end for}%
\newenvironment{myalgorithm}
{\rule{\textwidth}{0.5mm}\\\SetAlCapSty{}\SetAlgoNoEnd\SetAlgoNoLine\begin{algorithm}}{\end{algorithm}\rule{\textwidth}{0.5mm}}


%---------------------
\overfullrule=2mm
\pagestyle{plain}
\textwidth=15cm \textheight=685pt \topmargin=-25pt \linespread{1.3} 
\setlength{\parskip}{0pt}
\setlength\arraycolsep{2pt}
\oddsidemargin = 0.9cm
\evensidemargin =-0.1cm

\captionsetup{width=.95\linewidth, justification=centering}
%---------------------




\newtheorem{tw}{Twierdzenie}[section]
\newtheorem{lem}[tw]{Lemat}
\newtheorem{co}[tw]{Wniosek}
\newtheorem{prop}[tw]{Stwierdzenie}
\theoremstyle{definition}
\newtheorem{ex}{Przykład}
\newtheorem{re}[tw]{Uwaga}
\newtheorem{de}{Definicja}[section]



\newcommand{\bC}{{\mathbb C}}
\newcommand{\bR}{{\mathbb R}}
\newcommand{\bZ}{{\mathbb Z}}
\newcommand{\bQ}{{\mathbb Q}}
\newcommand{\bN}{{\mathbb N}}
\newcommand{\captionT}[1]{\caption{\textsc{\footnotesize{#1}}}}
\renewcommand\figurename{Rys.}

\numberwithin{equation}{section}
\renewcommand{\thefootnote}{\arabic{footnote})}
%\renewcommand{\thefootnote}{\alph{footnote})}



\begin{document}

% --------------------------------------------
% Strona tytułowa
% --------------------------------------------

\thispagestyle{empty}
\begin{titlepage}
\begin{center}\Large
Uniwersytet Komisji Edukacji Narodowej w Krakowie\\
\large
Instytut Bezpieczeństwa i Informatyki\\
\vskip 10pt
\end{center}
\begin{center}
\centering \includegraphics[width=1.0\columnwidth]{images/logo.png}
\end{center}

\begin{center}
 {\bf \fontsize{14pt}{14pt}\selectfont PROJEKT INŻYNIERSKI \\ DOKUMENTACJA PROJEKTOWA}
\end{center}
\vskip 5pt
\begin{center}
 {\bf \fontsize{22pt}{22pt}\selectfont TYTUŁ PROJEKTU}
\end{center}



\begin{center}
 {\fontsize{12pt}{12pt}\selectfont wykonany przez: }
\end{center}
\begin{center}
 {\bf\fontsize{16pt}{16pt}\selectfont Jan Kowalski}\\
 {\fontsize{12pt}{12pt}\selectfont Nr albumu: XXXXX \\\&\\}
 {\bf\fontsize{16pt}{16pt}\selectfont Anna Nowak}\\
 {\fontsize{12pt}{12pt}\selectfont Nr albumu: XXXXX\\\&\\}
 {\bf\fontsize{16pt}{16pt}\selectfont Karol Woźniak}\\
 {\fontsize{12pt}{12pt}\selectfont Nr albumu: XXXXX}
\end{center}
\begin{center}
 {\fontsize{12pt}{12pt}\selectfont pod opieką:}\\
 {\bf\fontsize{12pt}{12pt}\selectfont tytuł, imię i nazwisko opiekuna projektu}
\end{center}

%\mbox{}
\vspace*{\fill}
%\vskip 50pt
\begin{center}
\large
Kraków \the\year\\
(ostatnia aktualizacja: \DTMcurrenttime,\;\today)
\end{center}
\end{titlepage}
\setcounter{page}{0} 
\newpage\null\thispagestyle{empty}
%\setcounter{page}{0} 
%\newpage
%\thispagestyle{empty}

\tableofcontents


\newpage

\section{Szczegółowa dokumentacja projektowa}
\textit{W zależności od specyfiki projektu! Wymienione niżej podpunkty mają charakter orientacyjny.}
\subsection{Projekt UML}
\textit{W szczególności: diagram klas, ew. np. przypadki użycia, diagramy sekwencji, czynności, stanów, obiektów/komponentów/pakietów itp.}
\begin{center}
\centering \includegraphics[width=1.0\columnwidth]{images/UML.jpg}
\end{center}

\subsection{Projekt bazy danych}

\subsubsection*{Tabela \texttt{Address}}
\textbf{Opis:} Przechowuje informacje o adresach użytkowników.
\begin{itemize}
    \item \texttt{street} : varchar(255) – ulica
    \item \texttt{city} : varchar(100) – miasto
    \item \texttt{postal\string_code} : varchar(20) – kod pocztowy
    \item \texttt{country} : varchar(100) – kraj
    \item \texttt{is\string_default} : bool – flaga domyślnego adresu
    \item \texttt{user\string_id} : bigint (klucz obcy do \texttt{User})
    \item \texttt{id} : integer (klucz główny)
\end{itemize}

\subsubsection*{Tabela \texttt{Cart}}
\textbf{Opis:} Przechowuje informacje o koszykach zakupowych.
\begin{itemize}
    \item \texttt{created\string_at} : datetime – data utworzenia koszyka
    \item \texttt{updated\string_at} : datetime – data ostatniej aktualizacji
    \item \texttt{user\string_id} : bigint (klucz obcy do \texttt{User})
    \item \texttt{id} : integer (klucz główny)
\end{itemize}

\subsubsection*{Tabela \texttt{CartItem}}
\textbf{Opis:} Przechowuje informacje o produktach w koszyku.
\begin{itemize}
    \item \texttt{quantity} : integer unsigned – ilość produktu
    \item \texttt{added\string_at} : datetime – data dodania produktu
    \item \texttt{cart\string_id} : bigint (klucz obcy do \texttt{Cart})
    \item \texttt{product\string_id} : bigint (klucz obcy do \texttt{Product})
    \item \texttt{id} : integer (klucz główny)
\end{itemize}

\subsubsection*{Tabela \texttt{Category}}
\textbf{Opis:} Przechowuje informacje o kategoriach produktów.
\begin{itemize}
    \item \texttt{name} : varchar(100) – nazwa kategorii
    \item \texttt{description} : text – opis kategorii
    \item \texttt{id} : integer (klucz główny)
\end{itemize}

\subsubsection*{Tabela \texttt{Category\string_parent}}
\textbf{Opis:} Definiuje relacje hierarchiczne między kategoriami.
\begin{itemize}
    \item \texttt{from\string_category\string_id} : bigint (klucz obcy do \texttt{Category})
    \item \texttt{to\string_category\string_id} : bigint (klucz obcy do \texttt{Category})
    \item \texttt{id} : integer (klucz główny)
\end{itemize}

\subsubsection*{Tabela \texttt{Conversation}}
\textbf{Opis:} Przechowuje informacje o rozmowach.
\begin{itemize}
    \item \texttt{created\string_at} : datetime – data utworzenia rozmowy
    \item \texttt{is\string_admin\string_conversation} : bool – flaga rozmowy administracyjnej
    \item \texttt{order\string_id} : bigint (klucz obcy do \texttt{Order})
    \item \texttt{id} : integer (klucz główny)
\end{itemize}

\subsubsection*{Tabela \texttt{Conversation\string_participants}}
\textbf{Opis:} Przechowuje uczestników rozmowy.
\begin{itemize}
    \item \texttt{conversation\string_id} : bigint (klucz obcy do \texttt{Conversation})
    \item \texttt{user\string_id} : bigint (klucz obcy do \texttt{User})
    \item \texttt{id} : integer (klucz główny)
\end{itemize}

\subsubsection*{Tabela \texttt{Message}}
\textbf{Opis:} Przechowuje wiadomości w rozmowach.
\begin{itemize}
    \item \texttt{content} : text – treść wiadomości
    \item \texttt{timestamp} : datetime – znacznik czasu wiadomości
    \item \texttt{is\string_read} : bool – flaga odczytania wiadomości
    \item \texttt{conversation\string_id} : bigint (klucz obcy do \texttt{Conversation})
    \item \texttt{sender\string_id} : bigint (klucz obcy do \texttt{User})
    \item \texttt{id} : integer (klucz główny)
\end{itemize}

\subsubsection*{Tabela \texttt{Order}}
\textbf{Opis:} Przechowuje informacje o zamówieniach.
\begin{itemize}
    \item \texttt{status} : varchar(20) – status zamówienia
    \item \texttt{previous\string_status} : varchar(20) – poprzedni status
    \item \texttt{created\string_at} : datetime – data utworzenia
    \item \texttt{total\string_amount} : decimal – łączna kwota
    \item \texttt{delivery\string_address\string_id} : bigint (klucz obcy do \texttt{Address})
    \item \texttt{user\string_id} : bigint (klucz obcy do \texttt{User})
    \item \texttt{payment\string_method\string_id} : bigint (klucz obcy do \texttt{PaymentMethod})
    \item \texttt{id} : integer (klucz główny)
\end{itemize}

\subsubsection*{Tabela \texttt{PaymentMethod}}
\textbf{Opis:} Przechowuje informacje o metodach płatności.
\begin{itemize}
    \item \texttt{payment\string_method} : varchar(20) – typ metody
    \item \texttt{card\string_number} : varchar(16) – numer karty
    \item \texttt{expiration\string_date} : varchar(5) – data ważności
    \item \texttt{cvv} : varchar(4) – kod CVV
    \item \texttt{blik\string_code} : varchar(6) – kod Blik
    \item \texttt{user\string_id} : bigint (klucz obcy do \texttt{User})
    \item \texttt{id} : integer (klucz główny)
\end{itemize}

\subsubsection*{Tabela \texttt{Product}}
\textbf{Opis:} Przechowuje informacje o produktach.
\begin{itemize}
    \item \texttt{name} : varchar(100) – nazwa produktu
    \item \texttt{brand} : varchar(100) – marka produktu
    \item \texttt{image} : varchar(100) – obraz produktu
    \item \texttt{description} : text – opis produktu
    \item \texttt{price} : decimal – cena produktu
    \item \texttt{average\string_rate} : decimal – średnia ocena
    \item \texttt{product\string_details} : text – szczegóły produktu
    \item \texttt{product\string_images\string_links} : text – linki do zdjęć
    \item \texttt{id} : integer (klucz główny)
\end{itemize}

\subsubsection*{Tabela \texttt{Order\string_products}}
\textbf{Opis:} Przechowuje informacje o produktach w zamówieniach.
\begin{itemize}
    \item \texttt{order\string_id} : bigint (klucz obcy do \texttt{Order})
    \item \texttt{product\string_id} : bigint (klucz obcy do \texttt{Product})
    \item \texttt{id} : integer (klucz główny)
\end{itemize}

\subsubsection*{Tabela \texttt{Product\string_categories}}
\textbf{Opis:} Przechowuje relacje między produktami a kategoriami.
\begin{itemize}
    \item \texttt{product\string_id} : bigint (klucz obcy do \texttt{Product})
    \item \texttt{category\string_id} : bigint (klucz obcy do \texttt{Category})
    \item \texttt{id} : integer (klucz główny)
\end{itemize}

\subsubsection*{Tabela \texttt{Product\string_liked\string_by}}
\textbf{Opis:} Przechowuje informacje o użytkownikach, którzy polubili produkty.
\begin{itemize}
    \item \texttt{product\string_id} : bigint (klucz obcy do \texttt{Product})
    \item \texttt{user\string_id} : bigint (klucz obcy do \texttt{User})
    \item \texttt{id} : integer (klucz główny)
\end{itemize}

\subsubsection*{Tabela \texttt{RecommendedProducts}}
\textbf{Opis:} Przechowuje listy rekomendowanych produktów.
\begin{itemize}
    \item \texttt{added\string_at} : datetime – data dodania listy
    \item \texttt{user\string_id} : bigint (klucz obcy do \texttt{User})
    \item \texttt{id} : integer (klucz główny)
\end{itemize}

\subsubsection*{Tabela \texttt{RecommendedProducts\string_products}}
\textbf{Opis:} Przechowuje produkty powi\k{a}zane z rekomendacjami.
\begin{itemize}
    \item \texttt{RecommendedProducts\string_id} : bigint (klucz obcy do \texttt{RecommendedProducts})
    \item \texttt{product\string_id} : bigint (klucz obcy do \texttt{Product})
    \item \texttt{id} : integer (klucz główny)
\end{itemize}

\subsubsection*{Tabela \texttt{User\string_groups}}
\textbf{Opis:} Przechowuje relacje między użytkownikami a grupami.
\begin{itemize}
    \item \texttt{user\string_id} : bigint (klucz obcy do \texttt{User})
    \item \texttt{group\string_id} : integer (klucz obcy do \texttt{Auth\string_group})
    \item \texttt{id} : integer (klucz główny)
\end{itemize}

\subsubsection*{Tabela \texttt{UserCategoryVisibility}}
\textbf{Opis:} Przechowuje informacje o widoczności kategorii dla użytkowników.
\begin{itemize}
    \item \texttt{view\string_date} : datetime – data widoczności
    \item \texttt{category\string_id} : bigint (klucz obcy do \texttt{Category})
    \item \texttt{user\string_id} : bigint (klucz obcy do \texttt{User})
    \item \texttt{id} : integer (klucz główny)
\end{itemize}

\subsubsection*{Tabela \texttt{UserQueryLog}}
\textbf{Opis:} Przechowuje zapytania wykonane przez użytkowników.
\begin{itemize}
    \item \texttt{query} : varchar(255) – treść zapytania
    \item \texttt{query\string_date} : datetime – data zapytania
    \item \texttt{user\string_id} : bigint (klucz obcy do \texttt{User})
    \item \texttt{id} : integer (klucz główny)
\end{itemize}

\subsubsection*{Tabela \texttt{Profile}}
\textbf{Opis:} Przechowuje informacje o profilach użytkowników.
\begin{itemize}
    \item \texttt{last\string_opened\string_conversation\string_id} : bigint (klucz obcy do \texttt{Conversation})
    \item \texttt{user\string_id} : bigint (klucz obcy do \texttt{User})
    \item \texttt{id} : integer (klucz główny)
\end{itemize}

\subsubsection*{Tabela \texttt{Rate}}
\textbf{Opis:} Przechowuje oceny produktów wystawione przez użytkowników.
\begin{itemize}
    \item \texttt{value} : integer – wartość oceny
    \item \texttt{comment} : text – komentarz do oceny
    \item \texttt{created\string_at} : datetime – data wystawienia oceny
    \item \texttt{product\string_id} : bigint (klucz obcy do \texttt{Product})
    \item \texttt{user\string_id} : bigint (klucz obcy do \texttt{User})
    \item \texttt{id} : integer (klucz główny)
\end{itemize}

\subsubsection*{Tabela \texttt{Reaction}}
\textbf{Opis:} Przechowuje reakcje użytkowników na produkty.
\begin{itemize}
    \item \texttt{assigned\string_date} : datetime – data przypisania reakcji
    \item \texttt{type} : varchar(10) – typ reakcji (np. „like” lub „dislike”)
    \item \texttt{product\string_id} : bigint (klucz obcy do \texttt{Product})
    \item \texttt{user\string_id} : bigint (klucz obcy do \texttt{User})
    \item \texttt{id} : integer (klucz główny)
\end{itemize}

\subsubsection*{Tabela \texttt{UserProductVisibility}}
\textbf{Opis:} Przechowuje informacje o widoczności produktów dla użytkowników.
\begin{itemize}
    \item \texttt{view\string_date} : datetime – data widoczności
    \item \texttt{product\string_id} : bigint (klucz obcy do \texttt{Product})
    \item \texttt{user\string_id} : bigint (klucz obcy do \texttt{User})
    \item \texttt{id} : integer (klucz główny)
\end{itemize}

\subsubsection*{Tabela \texttt{UserRecommendedProductInteraction}}
\textbf{Opis:} Przechowuje informacje o interakcjach użytkowników z rekomendowanymi produktami.
\begin{itemize}
    \item \texttt{interaction\string_type} : varchar(20) – typ interakcji (np. kliknięcie, zakup)
    \item \texttt{interaction\string_date} : datetime – data interakcji
    \item \texttt{product\string_id} : bigint (klucz obcy do \texttt{Product})
    \item \texttt{user\string_id} : bigint (klucz obcy do \texttt{User})
    \item \texttt{id} : integer (klucz główny)
\end{itemize}

\subsubsection*{Tabela \texttt{User\string_user\string_permissions}}
\textbf{Opis:} Przechowuje informacje o uprawnieniach użytkowników.
\begin{itemize}
    \item \texttt{user\string_id} : bigint (klucz obcy do \texttt{User})
    \item \texttt{permission\string_id} : integer (klucz obcy do \texttt{Auth\string_permission})
    \item \texttt{id} : integer (klucz główny)
\end{itemize}

\subsubsection*{Tabela \texttt{User\string_query\string_log}}
\textbf{Opis:} Przechowuje zapytania wyszukiwania wykonane przez użytkowników.
\begin{itemize}
    \item \texttt{query} : varchar(255) – treść zapytania
    \item \texttt{query\string_date} : datetime – data zapytania
    \item \texttt{user\string_id} : bigint (klucz obcy do \texttt{User})
    \item \texttt{id} : integer (klucz główny)
\end{itemize}


\section*{Relacje między tabelami}
\begin{itemize}
    \item \texttt{Address.user\string_id} $\to$ \texttt{User.id}
    \item \texttt{Cart.user\string_id} $\to$ \texttt{User.id}
    \item \texttt{CartItem.cart\string_id} $\to$ \texttt{Cart.id}
    \item \texttt{CartItem.product\string_id} $\to$ \texttt{Product.id}
    \item \texttt{Category\string_parent.from\string_category\string_id} $\to$ \texttt{Category.id}
    \item \texttt{Category\string_parent.to\string_category\string_id} $\to$ \texttt{Category.id}
    \item \texttt{Conversation.order\string_id} $\to$ \texttt{Order.id}
    \item \texttt{Conversation\string_participants.conversation\string_id} $\to$ \texttt{Conversation.id}
    \item \texttt{Conversation\string_participants.user\string_id} $\to$ \texttt{User.id}
    \item \texttt{Message.conversation\string_id} $\to$ \texttt{Conversation.id}
    \item \texttt{Message.sender\string_id} $\to$ \texttt{User.id}
    \item \texttt{Order\string_products.order\string_id} $\to$ \texttt{Order.id}
    \item \texttt{Order\string_products.product\string_id} $\to$ \texttt{Product.id}
    \item \texttt{Product\string_categories.product\string_id} $\to$ \texttt{Product.id}
    \item \texttt{Product\string_categories.category\string_id} $\to$ \texttt{Category.id}
    \item \texttt{Product\string_liked\string_by.product\string_id} $\to$ \texttt{Product.id}
    \item \texttt{Product\string_liked\string_by.user\string_id} $\to$ \texttt{User.id}
    \item \texttt{RecommendedProducts\string_products.recommendedproducts\string_id} $\to$ \texttt{RecommendedProducts.id}
    \item \texttt{RecommendedProducts\string_products.product\string_id} $\to$ \texttt{Product.id}
    \item \texttt{User\string_groups.user\string_id} $\to$ \texttt{User.id}
    \item \texttt{UserCategoryVisibility.category\string_id} $\to$ \texttt{Category.id}
    \item \texttt{Profile.last\string_opened\string_conversation\string_id} $\to$ \texttt{Conversation.id}
    \item \texttt{Rate.product\string_id} $\to$ \texttt{Product.id}
    \item \texttt{Rate.user\string_id} $\to$ \texttt{User.id}
    \item \texttt{Reaction.product\string_id} $\to$ \texttt{Product.id}
    \item \texttt{Reaction.user\string_id} $\to$ \texttt{User.id}
    \item \texttt{UserRecommendedProductInteraction.product\string_id} $\to$ \texttt{Product.id}
    \item \texttt{UserRecommendedProductInteraction.user\string_id} $\to$ \texttt{User.id}
\end{itemize}


\subsubsection*{Przykładowe procedury składowane}

\subsubsection*{Dodawanie nowego użytkownika}
\begin{verbatim}
CREATE OR REPLACE PROCEDURE add_user(
    p_username VARCHAR,
    p_email VARCHAR,
    p_password VARCHAR
)
BEGIN
    INSERT INTO User (username, email, password)
    VALUES (p_username, p_email, p_password);
END;
\end{verbatim}

\subsubsection*{Pobieranie zamówień użytkownika}
\begin{verbatim}
CREATE OR REPLACE FUNCTION get_user_orders(p_user_id INT)
RETURNS TABLE(order_id INT, order_date DATETIME, status VARCHAR)
BEGIN
    RETURN QUERY
    SELECT id, order_date, status
    FROM Order
    WHERE user_id = p_user_id;
END;
\end{verbatim}


\subsection{Szczegółowa dokumentacja kodu}
\textit{Między innymi:}
\begin{itemize}
\item \textit{opis najważniejszych zmiennych;}
\item \textit{specyfikacja i opis wszystkich klas (jeśli projekt obiektowy) - opis całej klasy, jej pól i metod (jak poniżej);}
\item \textit{opis funkcji oraz metod klas (co robią, opis poszczególnych parametrów wejściowych i zwracanych wartości itp.) oraz w przypadku bibliotek programistycznych - przykłady użycia (przykładowy kod);} 
\item \textit{opis użytych wzorców projektowych.}
\end{itemize}
\subsection{Środowisko programistyczne}
\textit{Opis instalacji i konfiguracji niezbędnego środowiska programistycznego potrzebnego do ewentualnej dalszej pracy deweloperskiej z projektem (system operacyjny, wszelkie niezbędne narzędzia, biblioteki itp. wraz z sugerowanymi/minimalnymi ich wersjami).}





\renewcommand\refname{Literatura (jeżeli wymagana)}
\bibliography{references}
\addcontentsline{toc}{section}{Literatura}
% --------------------------------------------------------------------
%%%%%%% odkomentować gdy bibliografia ma być wewnątrz dokumentu
% --------------------------------------------------------------------
%\begin{thebibliography}{11}
%
%\addcontentsline{toc}{section}{Literatura}
%
%\bibitem{ZAN}
%C. Zannoni and P. Pasini, 
%\emph{Advances in the Computer Simulatons of Liquid Crystals}, Kluwer Academic Publishers, 2000.
%
%\end{thebibliography}

\end{document}

